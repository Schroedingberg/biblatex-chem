\documentclass[a4paper]{ltxdoc}
\usepackage{csquotes,lmodern,mathpazo,courier,helvet,multicol}
\usepackage[style=chem-rsc,hyperref]{biblatex}
\usepackage{hypdoc}

\bibliography{biblatex-chem}

\makeatletter
\renewcommand*\BibTeX{BibTeX}
\newcommand*\DescribeOption[1]{%
  \leavevmode
  \@bsphack
  \marginpar{\texttt{#1}}%
  \@esphack
  \ignorespaces
}
\providecommand*\pkg[1]{\textsf{#1}}
\makeatother

\RecordChanges

\title{The \textsf{biblatex-chem} bundle --- 
  \textsf{biblatex} styles for chemistry%
  \thanks{Describes version v1.0d, released 2011/01/17}}
\author{Joseph Wright%
  \thanks{E-mail: joseph.wright@morningstar2.co.uk}}
\date{Released 2011/01/17}
\begin{document}
\maketitle
\begin{abstract}
The \pkg{biblatex-chem} bundle is a set of styles for 
creating bibliographies using \pkg{biblatex} in the style of
a number common chemistry journals.  The bundle comprises styles
based on the conventions of the Royal Society of Chemistry, 
American Chemical Society and \emph{Angewandte Chemie}. It
therefore covers the journal styles of, for example:
\begin{itemize}
  \item \emph{Angewandte Chemie}
  \item \emph{Biochemistry}
  \item \emph{Chemical Communications}
  \item \emph{Chemistry---A European Journal}
  \item \emph{Dalton Transactions}
  \item \emph{Journal of the American Chemical Society}
  \item \emph{Organic \& Biomolecular Chemistry}
\end{itemize}
amongst others.
\end{abstract}

\begin{multicols}{2}
  \tableofcontents
\end{multicols}

\section{Introduction}
The \pkg{biblatex} package introduces a completely new method
for controlling the creation of bibliographies using
\BibTeX. This makes a great deal of flexibility 
available when creating bibliographies, most of which is much 
more difficult with traditional \BibTeX\ styles.

In order to use \pkg{biblatex}, an entirely new set of 
appropriate supporting styles are needed. This bundle provides a
number of styles for chemistry, following the rules of some of 
the most important journals in the field. 

\section{The styles}

Currently, the bundle contains four \pkg{biblatex} styles:
\begin{itemize}
  \item \texttt{chem-acs}: American Chemical Society style
  \item \texttt{chem-angew}: \emph{Angewandte Chemie} style
  \item \texttt{chem-biochem}: \emph{Biochemistry} style
  \item \texttt{chem-rsc}: Royal Society of Chemistry style
\end{itemize}
These can all be loaded in the usual way:
\begin{verbatim}
\usepackage[style=chem-<name>]{biblatex}
\end{verbatim}

The four styles can be used to follow the current layout rules
of all of the journals currently published by the American 
Chemical Society and the Royal Society of Chemistry, plus 
the journals published by Wiley which use the \emph{Angewandte
Chemie} format.

The styles use the standard \pkg{biblatex} database 
requirements, although they are somewhat selective in which
fields are used (for example, \texttt{language} is never 
printed).  This means that a database designed for traditional
\pkg{biblatex} use may need some editing for optimal output.
The accompanying example database \texttt{biblatex-chem.bib} 
shows examples of all of the supported entry types with common
fields filled in.

\section{Style options}

All of the styles here add a small number of package options to 
the standard set provided by \pkg{biblatex}. This allows 
the styles to cover the variations seen between different 
journals without needing a very large number of files: the 
American Chemical Society in particular varies the exact details
between journals.

\DescribeOption{articletitle}
The \texttt{articletitle} option is a switch for including the 
title of journal articles in the output. For the styles 
\texttt{chem-acs}, \texttt{chem-angew} and \texttt{chem-rsc} this 
is turned off on loading, and you therefore need to use for 
example
\begin{verbatim}
\usepackage[
  style=chem-rsc,
  articletitle
  ]{biblatex}
\end{verbatim}
or
\begin{verbatim}
\usepackage[
  style=chem-rsc,
  articletitle=true
  ]{biblatex}
\end{verbatim}
to include titles.  In the case of the \texttt{chem-biochem} 
style, the titles are included as standard: this can be turned
off using
\begin{verbatim}
\usepackage[
  style=chem-biochem,
  articletitle=false
  ]{biblatex}.
\end{verbatim}

\DescribeOption{biblabel}
The format of the numbers used in the bibliography varies from
journal to journal even if the same general style is used.  The
\texttt{biblabel} option allows the user to easily set the 
format used.  This option takes a value from the list: 
\texttt{parens}, \texttt{brackets}, \texttt{plain} and 
\texttt{dot}.
For example
\begin{verbatim}
\usepackage[
  style=chem-angew,
  biblabel=plain
  ]{biblatex}
\end{verbatim}
would use the style of \emph{Angewandte Chemie} but with nothing
added to the numbers in the bibliography.

\DescribeOption{chaptertitle}
The inclusion of chapter titles in \texttt{inbook} and 
\texttt{incollection} entries is not consistent in chemistry 
journals; the \emph{ACS Style Guide} recommends their use, but
does not require it \autocite{Coghill2006}. As a result, the
styles here provide the \texttt{chaptertitle} option to allow
the inclusion of chapter titles if required.
\begin{verbatim}
\usepackage[
  style=chem-rsc,
  chaptertitle
  ]{biblatex}
\end{verbatim}

\DescribeOption{doi}
When a journal article has no pages but does have a DOI the later used
in the bibliography. Some journal styles require that the DOI is 
included in all cases. This behaviour can be turned on by setting the
\texttt{doi} option.
\begin{verbatim}
\usepackage[
  style=chem-angew,
  doi
  ]{biblatex}
\end{verbatim}

\DescribeOption{pageranges}
Some journals only allow the first page of a range to be used, even
if the data for the full range is available. The \texttt{pageranges}
option can be used to turn off printing of the range for these
journals.
\begin{verbatim}
\usepackage[
  style=chem-rsc,
  pageranges=false
  ]{biblatex}
\end{verbatim}

\DescribeOption{subentry}
In common with the standard \pkg{biblatex} numeric styles, 
all of the styles here support the \texttt{subentry} option. 
With this turned on, entries of type \texttt{set} are given 
individual labels within the bibliography.  
\begin{verbatim}
\usepackage[
  style=chem-acs,
  subentry
  ]{biblatex}
\end{verbatim}
The result is very similar to the behaviour of the 
\textsf{mciteplus} package.

Different ACS journals use varying combinations of bibliography
styles (\texttt{chem-acs} or \texttt{chem-biochem}), citations 
styles, bibliography labels and article title inclusion.  In 
general, the user will need to consult a recent issue of the 
journal concerned to check which options to use.

\section{New styles}

The current set of styles here is intended to form a strong base
for chemists.  However, there will be the need for other styles
to be created. The package author welcomes suggestions for other
styles for inclusion.  It would also be good to keep all 
chemistry-related \pkg{biblatex} styles in one bundle. Others
working on chemistry styles for \pkg{biblatex} are welcome to
send them to the bundle maintainer so they can be incorporated 
here.

\section{Errors and omissions}

Creating a \pkg{biblatex} style is a complex task, 
particularly as chemistry journals only give complete rules
for the layout of references to journal articles.  The 
examples and test database provided with this bundle should be
consulted to see how the formatting is intended to work.  
Feedback on errors an omissions in either the database or the 
styles is very welcome. Please report any bugs either 
by e-mail 
\href{mailto:joseph.wright@morningstar2.co.uk}%
{joseph.wright@morningstar2.co.uk} or by filing a bug
in the 
\href{http://bitbucket.org/josephwright/biblatex-chem/issues}%
{BibBucket database}

\printbibliography

\changes{v1.0}{2010/11/20}{First stable release}
\changes{v1.0a}{2010/12/22}{Format \enquote{\emph{et al.}}~in italics
  when using \texttt{chem-rsc} style}
\changes{v1.0b}{2011/01/10}{Require \pkg{biblatex} v1.1}
\changes{v1.0b}{2011/01/10}{Use new \texttt{maxbibnames} option such
  that bibliographies print all authors but citations use truncated
  lists when necessary}
\changes{v1.0c}{2011/01/11}{Add version history for stable releases}
\changes{v1.0d}{2011/01/17}{Corrections for formatting of
  optionally-included article and chapter titles}
\changes{v1.0d}{2011/01/17}{Include additional punctuation tracker
  corrections for non-English bibliographies}

\PrintChanges

\end{document}
