%
% This file presents the `chem-biochem' style
%
\documentclass[a4paper]{article}
\usepackage[T1]{fontenc}
\usepackage[english,UKenglish,american]{babel}
\usepackage[babel]{csquotes}
\usepackage[
  style=chem-biochem,
  % articletitle,    % To include article titles
  % biblabel=dot,    % Alter bibliography labels
  % chaptertitle,    % Include chapter titles for parts of books
  % pagerange=false, % Only inlcude first page of a range
  % subentry,        % For (a), (b), etc. in sets
  hyperref
  ]{biblatex}
\usepackage[
  colorlinks,
  linkcolor=black,
  urlcolor=black,
  citecolor=black
  ]{hyperref}
\bibliography{biblatex-chem}

\begin{document}

\section*{The \texttt{chem-biochem} style}

This style prints numeric citations with bibliography
formatting following the rules of the American Chemical Society,
as implemented in the journal \emph{Biochemistry}. Citations 
will occur in-line, for example \autocite{Kabbe1973} or
\autocite{Arduengo1991}.

\nocite{*}

\printbibliography

\end{document}